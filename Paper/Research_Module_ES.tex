\documentclass[12pt,a4paper]{article}

%Standarteinstellungen
\usepackage[utf8]{inputenc}
\usepackage[T1]{fontenc}
\usepackage{tocloft}
\usepackage{lmodern}
\usepackage{ngerman}

%Mathtools
\usepackage{amsmath}
\usepackage{amsfonts}
\usepackage{amssymb}
\usepackage{amsthm}
\usepackage{mathrsfs}

%Graphiken/Tabellen
\usepackage{rotating}
\usepackage{multirow}
\usepackage{paralist}
\usepackage{graphicx}
\usepackage{braket}
\usepackage{bbm}

%Vordefinierte Umgebungen
\newtheorem{defi}{Definition}[subsection]
\newtheorem{satz}{Satz}[subsection]
\newtheorem{koro}{Korollar}[subsection]
\newtheorem{lemma}{Lemma}[subsection]
\newtheorem{prop}{Proposition}[subsection] 
\newtheorem{claim}{Claim}[subsection]
\newtheorem{thm}{Theorem}[subsection]
\newtheorem{coro}{Corollary}[subsection]

%Subfiles
\usepackage{subfiles}

%BibTeX
\usepackage{booktabs}              
\usepackage[authoryear]{natbib}			
\usepackage[nottoc]{tocbibind}			

%Layout
\usepackage[left=3.00cm, right=3.00cm, top=2.00cm, bottom=2.00cm]{geometry} 
\usepackage{setspace}
% anderthalbfacher Zeilenabstand 
\onehalfspacing
% kein Einrücken der ersten Zeile des Blocksatzes						
\parindent 0pt	
% keine Seitenzahl						
\pagenumbering{gobble}	
% für Punkte im Inhaltsverzeichnis 				
\renewcommand\cftsecleader{\cftdotfill{\cftdotsep}} 
%für römische Zahlen im Fließtext
\newcommand{\RM}[1]{\MakeUppercase{\romannumeral #1{}}}
%Für Programmcodes:
%Definieren uns farbigen Quellcode
\usepackage{color}
\definecolor{dkgreen}{rgb}{0,0.6,0}
\definecolor{gray}{rgb}{0.5,0.5,0.5}
\definecolor{mauve}{rgb}{0.58,0,0.82}
\newcommand{\grayScale}{0.95} 
\definecolor{codeBackground}{rgb}{\grayScale ,\grayScale ,\grayScale}
\definecolor{forestGreen}{rgb}{0.13,0.55,0.13}
%Damit wir Quellcode nutzen können.
\usepackage{listings}
\lstset{numbers=left,
	backgroundcolor=\color{codeBackground},
	frame=single,
	numberstyle=\tiny,
	numbersep=5pt,
	breaklines=true,
	rulecolor=\color{black},
	showstringspaces=false,
	xleftmargin=15pt,
	xrightmargin=15pt,
	basicstyle=\ttfamily\scriptsize,
	stepnumber=1,
	keywordstyle=\color{blue},         
	commentstyle=\color{dkgreen},       
	stringstyle=\color{mauve}         
}
%Sprache Festelegen
\lstset{language=R}
%Umlaute im Programmcode
\lstset{literate=%
	{Ö}{{\"O}}1
	{Ä}{{\"A}}1
	{Ü}{{\"U}}1
	{ß}{{\ss}}1
	{ü}{{\"u}}1
	{ä}{{\"a}}1
	{ö}{{\"o}}1
	{~}{{\textasciitilde}}1
}



\begin{document}
%Eigenständige Tex-Datei
\subfile{Titelpage.tex}
	
	
%Inhaltsverzeichnis	
%Unteren Befehl weglassen, dann ist es auf deutsch
\renewcommand{\contentsname}{Table of Contents}
\newpage 								
\tableofcontents	
					      



%Hier startet der generelle Inhalt	
\newpage
% mit Arabischen ziffern
\pagenumbering{arabic}

%Eigenständige Tex-Dateien zu jedem Kapitel
\subfile{Introduction}		

\subfile{Einfuehrung_CV}
\subfile{Linear_Setting}
\subfile{Leave_n_nu_out_CV}
\subfile{Model_selection_and_subclasses}	
\subfile{Simulations}




\section{Real Data}

\subsection{Comparison CV(1) and MCCV($n_{\nu}$): Prediction Error}


\subfile{Conclusion}

	
%Anhang 
\newpage
\pagenumbering{Roman} 		      % röm.ziffern	
\setcounter{page}{1}			  % ab ziffer 1

\addcontentsline{toc}{section}{Appendix}
\subfile{Appendix}



% Literaturverzeichnis
\newpage 								 
\medskip
%den unteren befehl herausnehmen, dann ist es auf deutsch
\renewcommand{\bibname}{References}
%Literaturlisten-stil
\bibliographystyle{myplainnat}
%LiteraturlisteRM_ES.bib erstellt und greift drauf zu        
\bibliography{LiteraturlisteRM_ES}   



%Schriftliche Versicherung
\newpage
\section*{Written insurance}
\addcontentsline{toc}{section}{Written insurance}


\begin{tabular}{lp{2em}l}
	%DATUM NOCHMAL PRÜFEN
	Rheinbach, den 17.07.2017   && 
	%UNTERSCHRIFT BEI PDF AKTIVIEREN
	%\includegraphics[width=0.15\textwidth,height=40px]{unterschrift.jpg}
	\\\cline{1-1}\cline{3-3}
	Ort, Datum     && Unterschrift
\end{tabular}


\end{document}