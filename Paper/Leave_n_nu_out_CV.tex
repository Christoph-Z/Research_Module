\documentclass[Research_Module_ES.tex]{subfiles}
\begin{document}
\section{The Leave-$n_\nu$-Out Cross-Validation}
%NOCH keine eigenen Worte (aus shao)
In this chapter we describe the \textit{Leave-$n_\nu$-Out Cross-Validation} by \cite{shao}. This Cross-Validation method does model selection by
\begin{align*}
CV(n_\nu)=\min_{\alpha\in\mathcal{A}}\hat{\Gamma}_{\alpha,n}^{CV}
\end{align*}
while $\hat{\Gamma}_{\alpha,n}^{CV}$ is the estimated \textit{Unconditional Expected Squared Prediction Error} $\Gamma_{\alpha,n}$. It yields an $\alpha^\star$ for getting the optimal model $\mathcal{M}_\star$.


First thing to do is splitting the given data into two parts, the \textit{set of validation} with $\{(y_i,x_i), i\in s\}$ and a \textit{training set} with $\{(y_i,x_i), i\in s^c\}$, where $s\subset\{1,...,n\}$. We have that $|s|\equiv n_\nu$ (number of datapoints used for validation) and $|s^c|=n-n_\nu$ (number of datapoints used for
%ANDers gehts glaube ich nicht ohne n_c wieder reinzunehmen (vektoren anders darstellen) (anndere wortwahl)
construction data) such that $n=|s|+|s^c|$. There are 
$\Big(\begin{matrix}
	n\\n_\nu
\end{matrix}\Big)$ different subsets $s$ with $n_\nu$ integers in it. For the estimator $\hat{\Gamma}_{\alpha,n}^{CV}$ we have to calculate the \textit{average squared prediction error of split $s$} under $\mathcal{M}_\alpha$. For this,   $\mathcal{M}_\alpha$ uses the construction data $\{(y_i,x_i), i\in s^c\}$. We know the definition of $ASPE(\mathcal{M}_\alpha)$ in (\ref{ASPE}). Thus for split $s$ we have a vector $y_s\in\mathbb{R}^{n_\nu}$ of future values out of validation set and the vector of the fitted value components under $\mathcal{M}_\alpha$, which is $\hat{y}_{\alpha,s^c}$ that uses the construction data. We have to average the squared prediction error by $|s|=n_\nu$ such that we achieve the following Lemma.
\begin{lemma}
	\label{Equation3.1}
	The average squared prediction error of split $s$ is given by
	\begin{align*}
		ASPE_{s}(\mathcal{M}_\alpha)&=\frac{1}{n_\nu}\parallel y_s-\hat{y}_{\alpha,s^c}\parallel_2^2\\
		&=\frac{1}{n_\nu}\parallel (I_{n_\nu}-Q_{\alpha,s})^{-1}(y_s-X_{\alpha,s}\hat{\beta}_\alpha)\parallel_2^2
	\end{align*}
\end{lemma}

\begin{proof}[Proof of Lemma \ref{Equation3.1}]~\\
	Given by construction, we have
	\begin{align*}
		X_\alpha=\Big(\begin{matrix}
		X_{\alpha,s}\\ X_{\alpha,s^c}
		\end{matrix}\Big)
		\text{~~~consisting of the two submatrices~} X_{\alpha,s}\in\mathbb{R}^{n_\nu\times d_\alpha} \text{~and~} X_{\alpha,s^c}\in\mathbb{R}^{(n-n_\nu)\times d_\alpha}
	\end{align*}
	and 
	\begin{align*}
		y=\Big(\begin{matrix}
		y_{s}\\ y_{s^c}
		\end{matrix}\Big),~~\hat{\beta}_\alpha=\Big(\begin{matrix}
		\hat{\beta}_{\alpha,s}\\ \hat{\beta}_{\alpha,s^c}
		\end{matrix}\Big)
	\end{align*}
	With the help of the following decompositions
	\begin{align*}
		X_\alpha^\prime y&=X_{\alpha,s}^\prime y_s+X_{\alpha,s^c}^\prime y_{s^c}\\
		X_\alpha^\prime X_\alpha&=X_{\alpha,s}^\prime X_{\alpha,s}+X_{\alpha,s^c}^\prime X_{\alpha,s^c}\\
		X_{\alpha,s} \hat{\beta}_{\alpha,s}&=P_{\alpha,s}y_s=I_{n_\nu}y_s=y_s
	\end{align*}
	Because $P_{\alpha,s}=P_{\alpha,s}^\prime P_{\alpha,s}=P_{\alpha,s}^{-1} P_{\alpha,s}=I_{n_\nu}$ is symmetric, idempotent and orthogonal. Thus now 
	we can rewrite $\hat{\beta}_\alpha$ as
	\begin{align}
		\hat{\beta}_\alpha&=(X_\alpha^\prime X_\alpha)^{-1}X_\alpha^\prime y   \nonumber\\
		&=(X_\alpha^\prime X_\alpha)^{-1}[X_{\alpha,s}^\prime y_s+X_{\alpha,s^c}^\prime y_{s^c}]  \nonumber \\
		&=(X_\alpha^\prime X_\alpha)^{-1}[(X_{\alpha,s}^\prime X_{\alpha,s}) \hat{\beta}_{\alpha,s}+(X_{\alpha,s^c}^\prime X_{\alpha,s^c})\hat{\beta}_{\alpha,s^c}]
		\label{beta_alpha_hat_decomposition}
	\end{align}
	The Average squared prediction error for split $s$ satisfies
	\begin{align*}
	ASPE_{s}(\mathcal{M}_\alpha)&=\frac{1}{n_\nu}\parallel y_s-\hat{y}_{\alpha,s^c}\parallel_2^2\\
	&=\frac{1}{n_\nu}\parallel y_s-X_{\alpha,s}\hat{\beta}_{\alpha,s^c}\parallel_2^2\\
	&=\frac{1}{n_\nu}\parallel (I_{n_\nu}-Q_{\alpha,s})^{-1}(I_{n_\nu}-Q_{\alpha,s})(y_s-X_{\alpha,s}\hat{\beta}_{\alpha,s^c})\parallel_2^2
	\end{align*}
	Thus it is enough to show that $(I_{n_\nu}-Q_{\alpha,s})(y_s-X_{\alpha,s}\hat{\beta}_{\alpha,s^c})=y_s-X_{\alpha,s}\hat{\beta}_\alpha$ holds 
	\begin{align*}
		&-Q_{\alpha,s}y_s-X_{\alpha,s}\hat{\beta}_{\alpha,s^c}+Q_{\alpha,s}X_{\alpha,s}\hat{\beta}_{\alpha,s^c}=-X_{\alpha,s}\hat{\beta}_\alpha\\
		\Leftrightarrow&-X_{\alpha,s}(X_{\alpha}^\prime X_{\alpha})^{-1}X_{\alpha,s}^\prime y_s- X_{\alpha,s}\hat{\beta}_{\alpha,s^c}+X_{\alpha,s}(X_{\alpha}^\prime X_{\alpha})^{-1}X_{\alpha,s}^\prime X_{\alpha,s}\hat{\beta}_{\alpha,s^c}=-X_{\alpha,s}\hat{\beta}_{\alpha}\\
		\Leftrightarrow&(X_{\alpha}^\prime X_{\alpha})^{-1}X_{\alpha,s}^\prime y_s+ \hat{\beta}_{\alpha,s^c}-(X_{\alpha}^\prime X_{\alpha})^{-1}X_{\alpha,s}^\prime X_{\alpha,s}\hat{\beta}_{\alpha,s^c}=\hat{\beta}_{\alpha}\\
		\Leftrightarrow&(X_{\alpha}^\prime X_{\alpha})^{-1}(X_{\alpha,s}^\prime X_{\alpha,s})\hat{\beta}_{\alpha,s}+ \hat{\beta}_{\alpha,s^c}-(X_{\alpha}^\prime X_{\alpha})^{-1}(X_{\alpha,s}^\prime X_{\alpha,s})\hat{\beta}_{\alpha,s^c}=\hat{\beta}_{\alpha}\\
		\Leftrightarrow&(X_{\alpha}^\prime X_{\alpha})^{-1}[(X_{\alpha,s}^\prime X_{\alpha,s})(\hat{\beta}_{\alpha,s}-\hat{\beta}_{\alpha,s^c})+(X_{\alpha}^\prime X_{\alpha})\hat{\beta}_{\alpha,s^c}]=\hat{\beta}_{\alpha}\\
		\Leftrightarrow&(X_{\alpha}^\prime X_{\alpha})^{-1}[(X_{\alpha,s}^\prime X_{\alpha,s})(\hat{\beta}_{\alpha,s}-\hat{\beta}_{\alpha,s^c})+[X_{\alpha,s}^\prime X_{\alpha,s}+X_{\alpha,s^c}^\prime X_{\alpha,s^c}]\hat{\beta}_{\alpha,s^c}]=\hat{\beta}_{\alpha}\\
		\Leftrightarrow&(X_\alpha^\prime X_\alpha)^{-1}[(X_{\alpha,s}^\prime X_{\alpha,s}) \hat{\beta}_{\alpha,s}+(X_{\alpha,s^c}^\prime X_{\alpha,s^c})\hat{\beta}_{\alpha,s^c}]=\hat{\beta}_{\alpha}
	\end{align*}
Which is exactly the decomposition of $\hat{\beta}_{\alpha}$ in   (\ref{beta_alpha_hat_decomposition}).\\
\end{proof}

Under the usage of Lemma \ref{Equation3.1} and averaging the $ASPE_{s}(\mathcal{M}_\alpha)$, we get the above mentioned estimator for $\Gamma_{\alpha,n}$.
\begin{lemma}
	For each model $\mathcal{M}_\alpha$, the estimate of $~\hat{\Gamma}_{\alpha,n}^{CV}$ of $~\Gamma_{\alpha,n}$ is obtained by
	\begin{align*}
		\hat{\Gamma}_{\alpha,n}^{CV}=\frac{1}{n \cdot n_\nu}\parallel (I_{n_\nu}-Q_{\alpha,s})^{-1}(y_s-X_{\alpha,s}\hat{\beta}_\alpha)\parallel_2^2
	\end{align*}
	
\end{lemma}


\subsection{Theory of CV(1) (Titel finden)}

\subsection{The Balanced Incomplete CV($n_\nu$)}

\subsection{The Monte Carlo CV($n_\nu$)}

\subsection{The Analytic Approximate CV($n_\nu$)}

\subsection{General Extension of CV($n_\nu$)}
A very helpful tool is the general extended version of the CV($n_\nu$) for different and more complicated types of initial models denoted as before by $\mathcal{M}_\alpha$. The general version under $\mathcal{M}_\alpha$ of this method of \cite{shao} does model selection by 
\begin{align*}
	GECV(n_\nu)=\min_{\alpha\in\mathcal{A}}\hat{\Gamma}_{\alpha,n}^{GECV}
\end{align*}
by the uncomplicate usage of the $GECV$-estimator for $\Gamma_{\alpha,n}$, denoted by
\begin{align*}
	\hat{\Gamma}_{\alpha,n}^{GECV}=\frac{1}{n_\nu b}\sum_{s\in\mathcal{L}}L(y_s,\hat{y}_{\alpha,s^c})
\end{align*} 
with $\mathcal{L}=\mathcal{B}$ for the $BICV(n_\nu)$ method and $\mathcal{L}=\mathcal{R}$ for the $MCCV(n_\nu)$ method. While $L(\cdot,\cdot)$ is the suitable loss function. As before, the data is splitted $b$ times, multiplied with the \textit{average squared prediction error} under $\mathcal{M}_\alpha$ for all splits $s\in\mathcal{L}$.
\end{document}