\documentclass[Research_Module_ES.tex]{subfiles}
\begin{document}
\subsection{The Monte Carlo CV($n_\nu$)}

%SHao vllt. irgendwo zitieren?
The \textit{Monte Carlo CV($n_\nu$)}, written in short notation as \textit{MCCV($n_\nu$)} is another variant of the \textit{CV($n_\nu$)} method and heavily related to the $BICV(n_v)$. It only distinguish to $BICV(n_v)$ by the way how it chooses his set of partitions. Indeed it does not choose them by a combinatorical procedure as described in the section before, this process is subjected to a random choice. Therefore it draws his subset of all possible partitions for a given $n_\nu$, which is denoted by $\mathcal{R}$, randomly with or without a replacement. $\mathcal{R}$ is the a collection of subsets such that $|\mathcal{R}|\equiv b$. This reminds to the $BICV(n_v)$, but without selecting $\mathcal{R}$ according to the \textit{balance conditions} . Therefore the \textit{MCCV($n_\nu$)} can be used as an alternative, if the collection of partitions $\mathcal{B}$ with its certain properties that have to be fullfilled is not provided. The \textit{MCCV($n_\nu$)} method does model selection by 
\begin{align*}
MCCV(n_\nu)=\min_{\alpha\in\mathcal{A}}\hat{\Gamma}_{\alpha,n}^{MCCV}
\end{align*}
where $\hat{\Gamma}_{\alpha,n}^{MCCV}$ is the estimator of $\Gamma_{\alpha,n-n_\nu}$. While the estimator is reffering again to Claim \ref{estimator CV(n_v)} and constructed similar to $\hat{\Gamma}_{\alpha,n}^{BICV}$ with the only difference that it uses $\mathcal{R}$ instead of $\mathcal{B}$. Thus no we get the following form
\begin{align*}
\hat{\Gamma}_{\alpha,n}^{MCCV}=|\mathcal{R}|^{-1}n_\nu^{-1}\sum_{s\in\mathcal{R}}\parallel y_s-\hat{y}_{\alpha,s^c}\parallel^2
\end{align*}
Now we consider the following Lemma\footnote{Proof of Lemma \ref{Lemma_MCCV} is given in Appendix \RM{1}}.
\begin{lemma}
	\label{Lemma_MCCV}
Denote by $\mathcal{R}^\ast:= \{s\subseteq\{1,\dots,n\}|\# s=n_v\}$ and let $\mathrm{E}_\mathcal{R}$ and $V_\mathcal{R}$ denote the expectation and variance with respect to the random selection of $\mathcal{R}\subseteq\mathcal{R}^\ast$. Then for any functions $a,b:\mathcal{R}^\ast\to \mathbb{R}$ it holds that
\begin{enumerate}
\item $\mathrm{E}_\mathcal{R} \bigl[ b^{-1}\sum_{s\in \mathcal{R}}a(s)\bigr] = \binom{n}{n_v}^{-1}\sum_{s\in\mathcal{R}^\ast}a(s)$
\item $V_\mathcal{R} \bigl( b^{-1}\sum_{s\in \mathcal{R}}a(s)\bigr) \le b^{-1} \mathrm{E}_\mathcal{R} \bigl[a(s)^2\bigr]$
\item $V_\mathcal{R}\bigl(a(s)+b(s)\bigr) \le 2\bigl[V_\mathcal{R}\bigl(a(s)\bigr)+V_\mathcal{R}\bigl(b(s)\bigr)\bigr]$
\item $\mathcal{R}^\ast$ is a balanced incomplete block design.
\end{enumerate}
\end{lemma}
Such that under the usage of this Lemma \ref{Lemma_MCCV} with the fact that $\mathcal{R}^\ast$ is a balanced incomplete block design, we can proof the following  Theorem\footnote{Proof of Theorem \ref{THM_Consistency_MCCV} is given in Appendix \RM{1}}. 


\begin{thm}[Consistency of $MCCV(n_v)$]
\label{THM_Consistency_MCCV}
Under the conditions of Theorem \ref{THM_Consistency of $CV(1)$} and
\begin{align*}
\max_{s\in \mathcal{R}}\biggl\lVert \frac{1}{n_v}\sum_{i\in s}x_ix_i' - \frac{1}{n-n_v}\sum_{i\in s^c}x_ix_i'\biggr\rVert =o_P(1)
\end{align*}
where $\mathcal{R}$ contains $b$ subsets selected randomly with $b$ satisfying
\begin{align*}
\frac{n^2}{b(n-n_v)^2}\to 0.
\end{align*}
Suppose furthermore that $n_v$ is selected such that
\begin{align*}
\frac{n_v}{n}\to 1 \quad \textrm{and} \quad n-n_v \to \infty.
\end{align*}
Then the following holds
\begin{enumerate}[(I)]
\item If $\mathcal{M}_\alpha$ is in Category I, then there exists $R_n \ge 0$ such that $\hat{\Gamma}_{\alpha,n}^{MCCV} = \frac{1}{n_vb}\sum_{s\in \mathcal{R}}\varepsilon_s'\varepsilon_s + \Delta_{\alpha,n} + R_n + o_P(1)$.
\item If $\mathcal{M}_\alpha$ is in Category II, then $\hat{\Gamma}_{\alpha,n}^{MCCV} = \frac{1}{n_vb}\sum_{s\in \mathcal{R}}\varepsilon_s'\varepsilon_s + \frac{1}{n-n_v}d_\alpha\sigma^2  + o_P((n-n_v)^{-1})$.
\item $\lim_{n\to\infty}P(\mathcal{M}_{MCCV}=\mathcal{M}_\ast) = 1$
\end{enumerate}
where $\mathcal{M}_{MCCV}$ denotes the model selected by using $MCCV(n_v)$.
\end{thm}

~\\\\~
\textbf{Überlegungen/Notizen}\\
- wie be BICV wahl von $n_\nu$ unter den oben genannten bedingungen ( sowie $n_\nu$ large and $(n-n\nu)$ small)\\
- zusätzlich, wenn $b$ die bedingung \begin{align*}
\frac{n^2}{b(n-n_v)^2}\to 0.
\end{align*}
erfüllt, dann auch hier MCCV besser als CV(1) weil consistent, je kleiner $(n-n_v)$ , desto mehr splits werden gebraucht, d.h. desto größer muss $b$ sein\\
- conditions
\begin{align*}
\frac{n_v}{n}\to 1 \quad \textrm{and} \quad n-n_v \to \infty.
\end{align*}
 und die obige bedingung 
 \begin{align*}
 \frac{n^2}{b(n-n_v)^2}\to 0.
 \end{align*}
 implizieren, dass $b\to \infty$ as $n\to\infty$



\end{document}