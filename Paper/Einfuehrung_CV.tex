\documentclass[Research_Module_ES.tex]{subfiles}

\begin{document}
\section{Cross Validation: General Framework}
The general Framework of cross validation was first described by \cite{stone1974cross} and simultaneously \cite{geisser1975predictive} as follows: \\

Suppose, we have data $Y=(y_1,\ldots,y_N)$ where each data point $y_i$ is associated with some $x_i$, i.e, $X=(x_1,\ldots,x_n)$. Note that $x$ and $y$ can be quite general.\\
\\
We are interested in predicting a future value of $y$ for a given value of $x$. Suppose we have a class of possible predictors 
\[
	\{\hat{y}=\hat{y}(Y,X,x,\alpha)|\alpha\in\mathscr{A}\}
\]
where $\alpha$ is some set of unknown values, e.g an underlying parametric model or a tuning parameter.\\
\\
The idea of cross validation is to choose some $\alpha^\star$ to minimize the loss between our prediction $\hat{y}$ and a corresponding real value of $y$. More explicitly we want $\alpha^\star$ to solve
\[
	\alpha^\star=\arg\min_{\alpha\in\mathscr{A}}\{L_n(Y,X\alpha)\}
\]
where $L_n$ is some loss function.
 
Since \cite{larson1931shrinkage} claimed that training an algorithm and evaluating it's statistical performance on the same data yields an overoptimistic result, we need to decompose our observations in order to train and evaluating our predictor. \\
\\
Denote $P^{(N-n)}_i$ to be the $i$'th partition of $Y$ and $X$ into $N-n$ retained and $n$ omitted observations, with $0<n<M$ where is $M$ the largest integer s.t $\hat{y}$ can be calculated, i.e,
\[
	P^{(N-n)}_i=(Y_{i,r}^{(N-n)},X_{i,r}^{(N-n)},Y_{i,o}^{(N-n)},X_{i,o}^{(N-n)})
\]
and let $\Gamma=\Gamma_{N-n}$ be the set of all relevant partitions of $N-n$ retained and $n$ omitted observations. Cross validation chooses $\alpha$ to minimize the average loss over the partitions in $\Gamma$ 
\[
	L_{N,n}^{CV}(\alpha)=P^{-1}n^{-1}\sum_{i\in\Gamma}L_n(Y_{i,o},\hat{Y})
\]
where $P^{-1}$ denotes the number of elements in $\Gamma$.\\
\\
The name cross validation refers to the way of splitting up the observations into several partition. The same procedure with just one partition of observations is called hold out or validation. Thus cross validation yields an averaging of several hold-out estimators.







\subsection{Grundlegende Einführung/Erklärung CV (Titel finden)}
\subsection{Motivation unserer Fragestellungen (Wie groß muss K gewählt werden, ist CV konsistent...)(Titel finden)}
\end{document}