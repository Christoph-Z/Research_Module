\documentclass[Research_Module_ES.tex]{subfiles}
\begin{document}
	
%--------------------Proof Uncobditional Expected Squared Error----------------------------

\begin{proof}[Proof undconditional expected squared error] ~\\
	Note that for randome variables $X$,$Z$ it holds true, that $E[E[Z|X,Z]|X]=E[Z|X]$
	\begin{align*}
		MSPE_{\alpha|X}=&E[MSPE_{\alpha|X,y}|X]\\
		=&E\left[\sigma^2 + n^{-1}\sum_{i}^{}\left(x_i^\prime\beta - x_{\alpha,i}\hat{\beta}_\alpha\right)\right]\\
		=&\sigma^2 +n^{-1}\sum_iE\left[\left(x_i^\prime\beta - x_{\alpha,i}^\prime\hat{\beta}_\alpha\right)^2\right]\\
		=&\sigma^2+n^{-1}\sum_iE\left[\left(x_i^\prime\beta-x_{\alpha,i}^\prime\hat{\beta}_\alpha\right)^\prime \left(x_i^\prime\beta-x_{\alpha,i}^\prime\hat{\beta}_\alpha\right)\right]\\
		=&\sigma^2 + n^{-1}\sum_iE\left[\beta^\prime x_ix_i^\prime\beta- \left(x_i^\prime\beta\right)^\prime x_{\alpha,i}^\prime\hat{\beta}_\alpha-\left(x_{\alpha,i}^\prime\hat{\beta}_\alpha \right)^\prime x_i^\prime\beta + \left(x_{\alpha,i}^\prime\hat{\beta}_\alpha\right)^\prime x_{\alpha,i}^\prime\hat{\beta}_\alpha \right]\\
		=&\sigma + n^{-1} \sum_i E\left[\beta^\prime x_ix_i^\prime\beta-\beta^\prime x_i x_{\alpha,i}^\prime\hat{\beta}_\alpha -\hat{\beta}_\alpha^\prime x_{\alpha,i}x_i^\prime\beta+\hat{\beta}_\alpha^\prime x_{\alpha,i}x_i^\prime\hat{\beta}_\alpha \right]\\
		=&\sigma^2+n^{-1}\sum_i\left\{ \beta^\prime x_ix_i^\prime\beta - E\left[ \beta^\prime x_i x_{\alpha,i}^\prime \left(X_\alpha^\prime X_\alpha\right)^{-1} X_\alpha^\prime y\right] \right. \\
		&\left.- E\left[y^\prime X_\alpha \left(X_\alpha^\prime X_\alpha\right)^{-1}x_{\alpha,i}x_i^\prime\beta\right] +E\left[y^\prime X_\alpha \left(X_\alpha^\prime X_\alpha\right)^{-1} x_{\alpha,i}x_{\alpha,i}^\prime \left(X_\alpha^\prime X_\alpha\right)^{-1} X_\alpha^\prime y\right] \right\}\\
		=&\sigma^2 + n^{-1}\sum_i\left\{\beta^\prime x_ix_i^\prime\beta- \beta^\prime x_i x_{\alpha,i}^\prime \left(X_\alpha^\prime X_\alpha\right)^{-1} X_\alpha^\prime X\beta -\beta^\prime X x_{\alpha,i} \left(X_\alpha^\prime X_\alpha\right)^{-1}x_{\alpha,i}x_i^\prime \beta \right.\\
		&+\left. E\left[ \beta^\prime X^\prime X_\alpha \left(X_\alpha^\prime X_\alpha\right)^{-1} x_{\alpha,i}x_{\alpha,i}^\prime \left(X_\alpha^\prime X_\alpha\right)^{-1} X_\alpha^\prime \left(X\beta +\epsilon\right)\right] \right.\\
		&+\left. E\left[\epsilon^\prime X_\alpha \left(X_\alpha^\prime X_\alpha\right)^{-1}x_{\alpha,i}x_{\alpha,i}^\prime \left(X_\alpha^\prime X_\alpha\right)^{-1} X_\alpha^\prime \left(X\beta +\epsilon\right) \right] \right\}\\
		=&\sigma^2 + n^{-1}\sum_i\left\{\beta^\prime x_ix_i^\prime\beta- \beta^\prime x_i x_{\alpha,i}^\prime 		\left(X_\alpha^\prime X_\alpha\right)^{-1} X_\alpha^\prime X\beta -\beta^\prime X x_{\alpha,i} \left(X_\alpha^\prime X_\alpha\right)^{-1}x_{\alpha,i}x_i^\prime \beta \right.\\
		&+\left.\beta^\prime X^\prime X_\alpha \left(X_\alpha^\prime X_\alpha\right)^{-1} x_{\alpha,i}x_{\alpha,i}^\prime \left(X_\alpha^\prime X_\alpha\right)^{-1} X_\alpha^\prime X\beta \right.\\ &+\left. E\left[\epsilon^\prime X_\alpha \left(X_\alpha^\prime X_\alpha\right)^{-1} x_{\alpha,i}^\prime \left(X_\alpha^\prime X_\alpha\right)^{-1} X_\alpha^\prime \epsilon^\prime \right]\right\}\\
		=&\sigma^2+ n^{-1}\left[\beta^\prime X^\prime X\beta-\beta^\prime X^\prime X_\alpha \left(X_\alpha^\prime X_\alpha\right)^{-1} X_\alpha^\prime X\beta \right]\\
		&-n^{-1}\left[\beta^\prime X^\prime X_\alpha \left(X_\alpha^\prime X_\alpha\right)^{-1} X_\alpha^\prime X\beta +\beta^\prime X^\prime X_\alpha \left(X_\alpha^\prime X_\alpha\right)^{-1} X_\alpha^\prime X_\alpha \left(X_\alpha^\prime X_\alpha\right)^{-1} X_\alpha^\prime X\beta\right]\\
		&+n^{-1}E\left[\epsilon^\prime X_\alpha \left(X_\alpha^\prime X_\alpha\right)^{-1} x_{\alpha,i}^\prime \left(X_\alpha^\prime X_\alpha\right)^{-1} X_\alpha^\prime \epsilon^\prime \right]\\
		=&\sigma^2+\Delta_{\alpha,n}+n^{-1}E\left[\epsilon^\prime X_\alpha \left(X_\alpha^\prime X_\alpha\right)^{-1} x_{\alpha,i}^\prime \left(X_\alpha^\prime X_\alpha\right)^{-1} X_\alpha^\prime \epsilon^\prime \right]\\
		=&\sigma^2+\Delta_{\alpha,n} + n^{-1}\sum_i\sum_j p_{\alpha,i,j}E\left[\epsilon_i\epsilon_j\right]\\
		=&\sigma^2+\Delta_{\alpha,n} + n^{-1}\sum_i p_{\alpha,i,i}\sigma^2\\
		=&\sigma^2+\Delta_{\alpha,n} + n^{-1}\text{tr}(P_\alpha)\sigma^2\\
		=&\sigma^2+\Delta_{\alpha,n} + n^{-1}d_\alpha\sigma^2
	\end{align*}
	Hence $\text{tr}(P_\alpha)=\text{tr}\left(X_\alpha \left(X_\alpha^\prime X_\alpha\right)^{-1} X_\alpha^\prime\right)=\text{tr}(X_\alpha^\prime X_\alpha \left(X_\alpha^\prime X_\alpha\right)^{-1})=\text{tr}(I_{d_\alpha})=d_\alpha $
\end{proof}


%-------------------------Proof for 2.5/2.5----------------------------------------

\begin{lemma}
	\label{Equation2.3-2.4}
	It holds true that:
	\begin{align*}
	&&\Delta_{\alpha,n}>0 && \text{,for } \mathcal{M}_\alpha\in\text{Category I}&&\\
	&&\Delta_{\alpha,n}=0 && \text{,for } \mathcal{M}_\alpha\in\text{Category II}&&
	\end{align*}
\end{lemma}
\begin{proof}[Proof Lemma \ref{Equation2.3-2.4}]~\\
	For the first part, assume $\mathcal{M}_\alpha\in$ Category I and note that we can rewrite $n\Delta_{\alpha,n}$ as follows
	\begin{align}\nonumber
	n\Delta_{\alpha,n}&= \left(X \beta \right)^\prime \left(I_n-P_\alpha\right)X\beta\\\nonumber
	&= \left(X \beta \right)^\prime \left(I_n-P_\alpha\right)^\prime\left(I_n-P_\alpha\right)X\beta\\\nonumber
	&=||\left(I_n-P_\alpha\right)X\beta||^2\\\nonumber
	&=||\left(I_n-P_\alpha\right)(X_\alpha\beta_\alpha+X_{\alpha^c}\beta_{\alpha^c})||^2\\
	&=||\left(I_n-P_\alpha\right)X_{\alpha^c}\beta_{\alpha^c}||^2>0 \label{larger0}
	\end{align}
	Hence 
	%\[
		%(\mathcal{M}_\alpha\in \text{ Cat I } \wedge \text{ r}(X)=p) \Rightarrow(\exists\beta_i\in \alpha^c:\beta_i\neq0 \wedge X_{\alpha^c}\text{ r}(X_{\alpha^c})=p-d_\alpha)\Rightarrow X_{\alpha^c}\beta_{\alpha^c}\neq0
	%\]
	$\mathcal{M}_\alpha\in$ Category I, it must exist at least one none zero component in $\beta_{\alpha^c}$. Together with $X$ having full rank, we can conclude that $X_{\alpha^c}\beta_{\alpha^c}$ is always unequal to zero.\\ 
	Note further, that $(I_n-P_\alpha)$ is the Projection matrix onto the orthogonal complement of span$\{x_{\alpha,1},\ldots,x_{\alpha,d_\alpha}\}$. Since
	\begin{align*}
		X_{\alpha^c}\beta_{\alpha^c}&\neq 0\\
		X_{\alpha^c}\beta_{\alpha^c}\not\perp (&I_n-P_\alpha)
	\end{align*}
	it holds that $(I_n-P_\alpha) X_{\alpha^c}\beta_{\alpha^c}\neq 0$ and therefore (\ref{larger0}) remains true.\\
	\\
	For the second part, assume $\mathcal{M}_\alpha\in$ Category II, then $X\beta=X_\alpha\beta_\alpha$. Thus
	\begin{align*}
		n\Delta_\alpha,n&=\beta^\prime X^\prime X\beta-\beta^\prime X^\prime X_\alpha \left(X_\alpha^\prime X_\alpha\right)^{-1}X_\alpha^\prime X\beta\\
		&=\beta_\alpha^\prime X_\alpha^\prime X_\alpha \beta_\alpha -\beta_\alpha^\prime X_\alpha^\prime X_\alpha\left(X_\alpha^\prime X_\alpha\right)^{-1} X_\alpha^\prime X_\alpha\beta_\alpha\\
		&=\beta_\alpha^\prime X_\alpha^\prime X_\alpha \beta_\alpha - \beta_\alpha^\prime X_\alpha^\prime X_\alpha \beta_\alpha\\
		&=0
	\end{align*}
	
\end{proof}



\end{document}