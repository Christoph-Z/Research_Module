\documentclass[Research_Module_ES.tex]{subfiles}
\begin{document}
\section{Introduction}	


















Cross validation is a technique to select models due to their predictive ability. Therefore a dataset of $n$ observations is split into $n-n_v$ data points for fitting the model and $n_v$ data points to obtain it's predictive ability.  \\
\\
Because of a low computational complexity the most popular variation of Cross validation is the leave one out Cross Validation, where $n_\nu$=1.
This is problematic, since one can show that this variation of Cross Validation is asymptotically incorrect. We will  show that this incorrectness can be rectified by using another $n_v$ which fulfils certain properties. 




\end{document}