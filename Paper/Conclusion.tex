\documentclass[Research_Module_ES.tex]{subfiles}
\begin{document}
\section{Conclusion}
As we have seen in our simulation examples, the \textit{MCCV$(n_\nu)$} version of \cite{shao} behaves much better than the asymptotically incorrect \textit{CV$(1)$}. So we can verify the theoretical part, which shows that the different \textit{CV$(n_\nu)$} methods improve over \textit{CV$(1)$}. Therefore we choosed $n_\nu$ with the heuristically recommended size of $n-\lfloor
n^{3/4}\rfloor$. We have also seen that the different variants behave better than \textit{AIC} which was not unexpected, because the \textit{AIC} and \textit{CV$(1)$} are asymptotically equivalent (this can be also seen in the simulation). A disadvantage of the \textit{CV$(n_\nu)$} methods is the computational complexitiy that increases by $n_\nu$. So we improved the code efficiency by redundant operations to get a faster calculation. Due to this computational complexity it is understandable that most authors prefer the much simpler but incorrect \textit{CV$(1)$} method. But in total we can see that the results of using the \textit{CV$(n_\nu)$} methods are better in comparison to the \textit{AIC} and \textit{CV$(1)$}. Thus we can accept a more complex calculation for a more precise outcome. 
\end{document}