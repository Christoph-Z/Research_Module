\documentclass[Research_Module_ES.tex]{subfiles}
\begin{document}
\subsection{The Analytic Approximate CV($n_\nu$)}
%NOTIZEN keine ahnung wie man das ausformulieren soll und am besten zeigen soll mit hilfe eines nachweises (evtl den special case aus shao zitieren und die umformungen die er gemacht hat falls wir es schaffen noch nachholen) sonst sieht der estimator aus als sei er aus der luft gegriffen
The third variant of CV($n_\nu$) we want to mention is the so called \textit{Analytic Approximate CV($n_\nu$)} of \cite{shao}. We will write it for simplicity as \textit{APCV($n_\nu$)}. This method does also model selection by 
\begin{align*}
APCV(n_\nu)=\min_{\alpha\in\mathcal{A}}\hat{\Gamma}_{\alpha,n}^{APCV}
\end{align*}
The \textit{APCV($n_\nu$)} is constructed as an approximation of the \textit{BICV($n_\nu$)}. In the proof of Theorem \ref{THM_Consistency_BICV} we derive the following asymptotic representation for the \textit{BICV($n_\nu$)} estimate
\footnote{See equations (\ref{motivation_APCV_1}),(\ref{motivation_APCV_2}),(\ref{proof_BICV_4}) and (\ref{proof_BICV_6}).}
\begin{align*}
\hat{\Gamma}_{\alpha,n}^{BICV}=\frac{1}{n}\lVert y-X_\alpha\hat{\beta}_\alpha\rVert^2 + \frac{2n-n_v}{(n-n_v)(n-1)}\sum_{i=1}^np_{i,i,\alpha}e_i^2 + B_\alpha,
\end{align*}
where $B_\alpha = o_P((n-n_\nu)^{-1})$. 
%Note that the first part of this representation doesn't depend on $\mathcal{B}$.
Motivated by this representation, $\hat{\Gamma}_{\alpha,n}^{APCV}$ is chosen to be
\begin{align*}
\hat{\Gamma}_{\alpha,n}^{APCV}=\frac{1}{n}\lVert y-X_\alpha\hat{\beta}_\alpha\rVert^2 + \frac{2n-n_v}{(n-n_v)(n-1)}\sum_{i=1}^np_{i,i,\alpha}e_i^2,
\end{align*}
i.e. we omit the remainer term $B_\alpha$.
Note that the computation of $\hat{\Gamma}_{\alpha,n}^{APCV}$ does not rely on partitioning and therefore the \textit{APCV($n_\nu$)} requires less computations in comparison to \textit{BICV($n_\nu$)} and \textit{MCCV($n_\nu$)}.

Moreover, \cite{shao} shows that $B_\alpha$ vanishes whenever the following condition holds
\begin{align}
\label{Bedingung_BICV_example}
	\frac{1}{n_\nu}\sum_{i\in s}x_i x_i^\prime=\frac{1}{n-n_\nu}\sum_{i\in s^c}x_i x_i^\prime\quad \forall s\in\mathcal{B}
\end{align}
and hence $\hat{\Gamma}_{\alpha,n}^{APCV}=\hat{\Gamma}_{\alpha,n}^{BICV}$ if and only if condition (\ref{Bedingung_BICV_example}) is satisfied.
%As the \textit{MCCV($n_\nu$)}, we can make a relation to the \textit{BICV($n_\nu$)}. Under the
%SEite  489 in shao Gleichung (3.17) die unter der condition (3.16) erzeugt werde kann (und den bedingungen des examples ....)
%case of the example \cite{shao} has given in his part to the \textit{BICV($n_\nu$)}, where the condition that $\mathcal{B}$ can be chosen such that
%\begin{align}
%\label{Bedingung_BICV_example}
%	n_\nu^{-1}\sum_{i\in s}x_i x_i^\prime=(n-n_\nu)^{-1}\sum_{i\in s^c}x_i x_i^\prime~~~\forall s\in\mathcal{B}
%\end{align}
%holds, we can derive a special expression for the estimator of $\hat{\Gamma}_{\alpha,n}^{BICV}$. This special representation for $\hat{\Gamma}_{\alpha,n}^{BICV}$ under this certain condition is chosen to be the estimator $\hat{\Gamma}_{\alpha,n}^{APCV}$ and denoted by
%NAchweis wie er drauf kommt einfügen, falls noch zeit ist
%\begin{align*}
%\hat{\Gamma}_{\alpha,n}^{APCV}=\frac{1}{n}\lVert y-X_\alpha\hat{\beta}_\alpha\rVert^2 + \frac{2n-n_v}{(n-n_v)(n-1)}\sum_{i=1}^np_{i,i,\alpha}e_i^2
%\end{align*} 
%Thus only under (\ref{Bedingung_BICV_example}) we have that 
%$\hat{\Gamma}_{\alpha,n}^{APCV}=\hat{\Gamma}_{\alpha,n}^{BICV}$.\\\\
%We can show that the following Corollary holds.\footnote{Proof of Corollary \ref{Consistency_APCV} is given in Appendix \RM{1}}

Since $B_\alpha=o_P((n-n_\nu)^{-1})$, we expect that the \textit{APCV($n_\nu$)} inherits the asymptotic properties of \textit{BICV($n_\nu$)} and indeed, the following corollary proves this to be right.\footnote{Proof of Corollary \ref{Consistency_APCV} is given in Appendix \RM{1}}
\begin{coro}[Asymptotic Properties of \textit{APCV($n_v$)}]
\label{Consistency_APCV}
Under the conditions of Theorem \ref{THM_Consistency_BICV}, the statements $(I)-(III)$ hold for the Analytic Approximate $CV(n_v)$.
\end{coro}
%In comparison to \textit{BICV($n_\nu$)} and \textit{MCCV($n_\nu$)}, the \textit{APCV($n_\nu$)} requires less computations, since no partitioning is necessary. 
%Another important point is, that \textit{APCV($n_\nu$)} depends on the linear setting in contrast to the other procedures. Therefore it is not as easy to exend to other scenarios.
The derivation of \textit{APCV($n_\nu$)} depends on the linear setting in contrast to the other procedures. Therefore it is not as easy to exend to other scenarios.
Furthermore note that \textit{APCV($n_\nu$)} does not perform as well as the \textit{MCCV($n_\nu$)} in the simulation of \cite{shao}.
\end{document}